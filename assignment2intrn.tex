\documentclass[a4paper,12pt]{article}
\usepackage{amsmath}
\begin{document}
\title{Assignment 2}
\author{Junaid Ahmad Bhat}
\date{\today}
\maketitle
\section*{\small Question}
If A and B are square matrices of the same order such that AB=BA, then prove by induction that \textit{AB $^n$=B $^n$A.} \\ 
Further, prove that \textit{(AB)$^n$=A $^n$B $^n$} for all n$\epsilon$N.\\

\section*{\small Solution}

A and B are square matrices of the same order such that AB=BA\\

To prove: P(n):\textit{AB$^n$=B$^n$A},n$\epsilon$N\\

For n=1, we have:\\

P(1):\textit{AB=BA} [Given]\\

\textit{AB$^1$=B$^1$A}\\

i,e the result is true for n=1.\\

Let the result be true for n=k.\\

P(k):\textit{AB$^k$=B$^k$A}   \hspace{4cm}(1)\\

Now, we prove that the result is rue for n=k+1.\\

\textit{AB $^k$$^+$$^1$=AB$^k$.B=(B$^k$A)B} (By(1))\\

            \hspace{4cm}   =\textit{B$^k$(AB)} \hspace{2cm}(By Associative law)\\

              \hspace{4cm} =\textit{B$^k$(BA}) \hspace{2cm}(AB=BA (given))\\

               \hspace{4cm} =\textit{(B$^k$B)A} \hspace{2cm}( By Associative law)\\

                \hspace{4cm} =\textit{B $^k$$^+$$^1$A}\\

i,e the result is true for n=k+1.\\

Thus, by the principle of mathematical induction , 
we have \textit{AB $^n$=B$^n$A},n$\epsilon$N.\\

Now, we prove that \textit{(AB$^n$)=A $^n$B $^n$} for all n$\epsilon$N\\

For n=1, we have:\\

\textit{(AB) $^1$=A $^1$B $^1$=AB}\\

i,e the result is true for n=1.\\

Let it be true for n=k.\\

\textit{(AB) $^k$=A$^k$B$^k$}     \hspace{4cm}(2)\\

Now we prove that the result is true for n=k+1.\\

\textit{(AB)$^k$$^+$$^1$=(AB)$^k$.(AB)=(A $^k$B $^k$)(AB)}.\hspace{2cm} (By (2) )\\

            \hspace{4cm} \textit{=A$^k$ (B$^k$ A)B} \hspace{2cm}( By Associative law)\\

           \hspace{4cm} =\textit{A$^k$(AB$^k$)B}     \hspace{2cm}(AB$^n$=B$^n$A, n$\epsilon$N)\\

            \hspace{4cm}=\textit{(A$^k$A).(B$^k$B)} \hspace{2cm}(By Associative law)\\

            \hspace{4cm}=\textit{A$^k$$^+$$^1$B$^k$$^+$$^1$}\\ 

 

i,e the result is true for n=k+1.\\

Therefore, by the principle of mathematical induction , we have \textit{(AB)$^n$=A$^n$B$^n$},\\  

for all natural numbers.


\section*{{\small Question}} Let A=$\begin{pmatrix} 0 & 1  \\ 0 & 0 \end{pmatrix}$ ,show that \textit{(aI + bA)$^n$= a$^n$I+na$^n$$^-$$^1$bA},where I is tha identity matrix of order 2 and n$\epsilon$N.

\section*{{\small Solution}}Using principle of mathematical induction\\
For n=1,we have\\

\textit{P(1):(aI+bA)=aI+a$^0$bA=aI+bA}\\

i,e 
result is true for n=1\\

Let it be true for n=k\\
               
i,e
\textit{P(k):(aI+bA)$^k$=a$^k$I+ka$^k$$^-$$^1$bA}\\

Now,we have to prove the result is true for n=k+1.\\

\textit{(aI+bA)$^k$$^+$$^1$=(aI+bA)$^k$(aI+bA)}\\

               \hspace*{2.2cm}     \textit{=(a$^k$I+ka$^k$$^-$$^1$bA)(aI+bA)}\\

                 \hspace*{2.2cm}   \textit{=a$^k$$^+$$^1$I+ka$^k$bAI+a$^k$bIA+ka $^k$$^-$$^1$b$^2$A$^2$}\\

                 \hspace*{2.2cm}     \textit{=a$^k$$^+$$^1$I+(k+1)a$^k$bA+ka$^k$$^-$$^1$b$^2$A$^2$}         \hspace{3cm}(1)\\

Now,
A$^2$=$\begin{bmatrix} 0 & 1  \\ 0 & 0 \end{bmatrix}$ $\begin{bmatrix} 0 & 1  \\ 0 & 0 \end{bmatrix}$=$\begin{bmatrix} 0 & 0  \\ 0 & 0 \end{bmatrix}$=0\\

Therfore,eqn(1) becomes\\
\textit{(aI+bA)$^k$$^+$$^1$=a$^k$$^+$$^1$I+(k+1)a$^k$bA}\\

Hence proved result is true for n=k+1\\

Therefore, by principle of mathematical induction we have\\
\textit{(aI + bA)$^n$= a$^n$I+na$^n$$^-$$^1$bA}\\




\end{document}
